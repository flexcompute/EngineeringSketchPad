\documentclass[titlepage]{article}
\usepackage{graphicx}
\usepackage{pdfpages}
\DeclareGraphicsExtensions{.png}
\usepackage{geometry}
\geometry{verbose,tmargin=1in,bmargin=1in,lmargin=1in,rmargin=1in}

\title{\vspace{-1.0in}
       {\LARGE {\tt ESP}: \\ {\bf EGADS Interface to the {\bf C}himera {\bf G}rid {\bf T}ools}}\\
       \vspace{0.20in}
       Discrete Output of Analytic Geometry for the {\bf CGT} Suite}
\author{Bob Haimes\\
  Aerospace Computational Design Laboratory \\
  Department of Aeronautics \& Astronautics\\
  Massachusetts Institute of Technology\\
  haimes@mit.edu}
\date{15 November 2019}

\begin{document}

\let\originalnewpage\newpage
\let\newpage\relax
\maketitle
\let\newpage\originalnewpage

\vspace{0.5in}
\section{Introduction}

The goal of this work is to begin the automation of geometry import into the
{\bf C}himera {\bf G}rid {\bf T}ools, including {\tt OverGrid}. This will be accomplished by
an EGADS (part of the {\tt ESP} toolset) application that will take as input
an analytic geometry description (via IGES, STEP or native EGADS files).
The output will be an enhanced {\it tri} file, PLOT3D files and an ancillary
file that will contain all of the data necessary to reassociate the surface discretization
back to the appropriate geometric entities.

\vspace{0.075in}
This application (called {\tt egads2cgt}) will become part of the {\tt ESP} open source software suite
in the next release and will be distributed with {\tt ESP}. Users of {\bf CGT} will need to download and install {\tt ESP}
in order to connect {\tt egads2cgt} to the appropriate components of the {\bf C}himera {\bf G}rid {\bf T}ools.

\newpage
\section{Application}

\begin{verbatim}
  Usage: egads2cgt [argument list] (defaults are in parenthesis)
         -i  input geometry filename <*.egads, *.stp, *.step, *.igs, *.iges>
         -aflr4 use AFLR4 for surface triangulations (and not EGADS)
         -q  write structured patches from quadding scheme to plot3d surface grid file
         -uv write structured patches from uv evaluation to plot3d surface grid file
         -ggf factr <geometric growth factor with -uv for isocline smoothing> (1.2)
            
         -maxa mxang <Max allow dihedral angle (deg)>            (15.0)
         -maxe mxedg <Max allow relative edge length>            ( 0.025 * size)
         -maxc mxchd <Max allow relative chord-height tolerance> ( 0.001 * size)
\end{verbatim}

This application always generates {\tt Cart3D} {\it tri} files and {\tt egads2cgt} {\it tess} files.
There will be as many files of each type as there are EGADS bodies found in the input
geometry files.

\vspace{0.1in}
\noindent
Note: the scheme used to generate a UV mesh has been enhanced in order to better capture
regions of high curvature (in comparison to earlier apps -- {\tt egads2srf}).

\section{Outputs}

The files will have names consistent with the input filename (without the extension), then a ``.'' and 3 digits
to indicate the EGADS {\it Body} index and finally the extension. For example: {\tt fighter.001.a.tri}, {\tt fighter.001.p3d},
{\tt fighter.002.tess}.

\subsection{Cart3D {\it tri} file}

This is a standard Cart3D {\it tri} file with the exception of the use of the component field. Traditionally
this integer (per triangle) is used to separate the triangulation of one {\it solid} component from another (for
example: OML and stores). Since there will only be a single component per {\it tri} file, we will use this integer
to mark the Face ID where the triangle resides. Therefore this {\it tri} file should NOT be directly used with the Cart3D
software suite.

\subsection{PLOT3D surface file}

We will use standard PLOT3D surface files to output structured meshes of quadrilaterals. These will be generated
only when either the {\tt -q} or {\tt -uv} options are specified on the {\tt egads2cgt} command line.

\subsection{{\tt egads2cgt} {\it tess} vertex owner file}

This {\it new} file type holds all of the necessary data so that vertices in the {\it tri} file can be placed back on the
geometry with the appropriate parameters. Also this is where attributes (meta-data) that has been placed on geometric
entities are found. The file format is:

\begin{verbatim}
      bodyType  nNodes  nEdges  nLoops  nFaces
      
      [ i gIndex 
        nAttr
        [ attrRecs x nAttr ] ] x nNodes
      
      [ i nEPts
         [ gIndex, t, 0/nIndex ] x nEPts
         nAttr
         [ attrRecs x nAttr ]  ] x nEdges
         
      [ i nedge
        [ eIndex sense ] x nedge
        nAttr
        [ attrRecs x nAttr ]  ] x nLoops
      
      [ i nloop norF nFpts
         lIndex x nloop
         [ gIndex, u, v, type, index ] x nFpts
         nAttr
         [ attrRecs x nAttr ]  ] x nFaces
  
      
where: bodyType is 6 for WireBody, 7 for FaceBody, 8 for SheetBody, 9 for SolidBody

       gIndex  is the global Index in the tri file (bias 1)
       i       is the index of the entity (1 bias)
       nAttr   is the number of attributes found with the entity
       nNodes  is the number of Nodes in the Body
       nEdges  is the number of Edges in the Body
       nLoops  is the number of Loops in the Body
       nFaces  is the number of Faces in the Body
       
       nEpts   is the number of discretization points for the Edge 
                 (including Nodes at beginning and end) -- 0 degenerate
       t       is the Edge parameter value that corresponds to the vertex
       nIndex  is the Node Index (bias 1)
       
       nedge   is the number of Edges in the Loop
       eIndex  is the Edge Index (bias 1)
       sense   is -1 or 1 
       
       nloop   is the number of loops bounding the Face
       norF    is the Face Normal orientation (-1 or 1)
       nFpts   is the number of discretization points for the Face 
       lIndex  is the loop index that trims the Face
       u and v are the Face parameter values that correspond to the vertex
       type    is the type of the vertex [- is interior, 0 is a Node, + is on an Edge]
       index   is the Node/Edge index 
       
       attrRec is the attribute information in this form:
       
           [type len name]
           [data]
       
           where: type = 1 -- integers
                         2 -- doubles
                         3 -- character string
                  len  is the number of data values
                  data is the record containing the information in the form of type and
                       of the specified length
\end{verbatim}

\end{document}